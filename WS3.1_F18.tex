\documentclass[11pt]{article}  % This tells LaTex what sort of document style to use

\usepackage{amssymb}  % These lines tell LaTeX to load additional symbol definitions and style definitions
\usepackage{amsthm}
\usepackage{amsmath}
\usepackage{enumerate}
\usepackage{graphicx}
\usepackage{mathtools}
\usepackage{multicol}
\usepackage{parskip}
\usepackage{changepage}
\usepackage{setspace}
\usepackage{xcolor}
\usepackage{tikz,tikz-cd}	
\usetikzlibrary{positioning, matrix, shapes}  

%%%%%%%%%%%%%%%%%%%%%%%%%%%%%%%%%%%%%%%%%%%%%%
%Formatting
\setlength{\parskip}{0cm}
\setlength{\parindent}{1cm}
\singlespacing

\makeatletter
\renewcommand{\paragraph}{%
  \@startsection{paragraph}{4}%
  {\z@}{3.25ex \@plus 1ex \@minus .2ex}{-1em}%
  {\normalfont\normalsize\bfseries}%
}
\makeatother

\DeclarePairedDelimiter\abs{\Bigl\bigg|}{\Bigr\bigg|}
\DeclarePairedDelimiter\ceil{\lceil}{\rceil}

%%%%%%%%%%%%%%%%%%%%%%%%%%%%%%%%%%%%%%%%%%%%%%
%  First we set the page layout.

\usepackage[top=1 in, bottom=1in,left=.5in,right=.5in]{geometry}

%%%%%%%%%%%%%%%%%%%%%%%%%%%%%%%%%%%%%%%%%%%%%%
%  Now create some user defined commands.

\newcommand{\map}[1]{\xrightarrow{\#1}}
\newcommand{\iso}{\cong}
\newcommand{\define}{\stackrel{\mathrm{def}}{=}}

\newcommand{\N}{\mathbb N}
\newcommand{\Z}{\mathbb Z}
\newcommand{\Q}{\mathbb Q}
\newcommand{\R}{\mathbb R}
\newcommand{\C}{\mathbb C}
\newcommand{\e}{\epsilon}
\newcommand{\A}{\mathbb A}
\newcommand{\p}{\mathbb P}

\DeclareMathOperator{\lcm}{lcm}

%  End user defined commands
%%%%%%%%%%%%%%%%%%%%%%%%%%%%%%%%%%%%%%%%%%%%%%

%%%%%%%%%%%%%%%%%%%%%%%%%%%%%%%%%%%%%%%%%%%%%%
% These establish different environments for stating Theorems, Lemmas, Remarks, etc.

\newtheorem{Thm}{Theorem}
\newtheorem{Prop}[Thm]{Proposition}
\newtheorem{Lem}[Thm]{Lemma}
\newtheorem{Cor}[Thm]{Corollary}
\newtheorem{Axiom}[Thm]{Axiom}

\theoremstyle{definition}
\newtheorem{Def}[Thm]{Definition}
\newtheorem{Ex}[Thm]{Example}
\newtheorem{Exercise}[Thm]{Exercise}
\newtheorem{Fact}[Thm]{Fact}
\newtheorem{Claim}[Thm]{Claim}
\newtheorem*{Pf}{Proof}
\newtheorem{Prob}[Thm]{Problem}

\DeclareMathOperator{\spn}{span}

\theoremstyle{remark}
\newtheorem{Rem}[Thm]{Remark}

\newcounter{solution} \setcounter{solution}{1}
\newenvironment{Solution}[1][]{\noindent {\textbf{Solution
\arabic{solution}. #1} \stepcounter{solution} \rm}}{ \qed \newline}

\renewcommand{\labelenumi}{(\alph{enumi})}
\newcommand{\lam}{\lambda}

% End environments 
%%%%%%%%%%%%%%%%%%%%%%%%%%%%%%%%%%%%%%%%%%%%%%%

%%%%%%%%%%%%%%%%%%%%%%%%%%%%%%%%%%%%%%%%%%%%%%
% Now we're ready to start
%%%%%%%%%%%%%%%%%%%%%%%%%%%%%%%%%%%%%%%%%%%%%%

\begin{document}

\noindent {\bf \huge WES Worksheet 3.1} \qquad\qquad\qquad\qquad\qquad\qquad\qquad\qquad\qquad\qquad\qquad {\large Fall 2018}\\[.2cm]
{\large MATH 222, Week 3}\newline \newline
\noindent {\bf Name: \underline{\hspace{7cm}}}
\newline
\thispagestyle{empty}  
\pagestyle{plain}
\setcounter{page}{1}
\pagenumbering{arabic}

\section*{Factoring polynomials}
This section will focus on factoring polynomials. In general this is an incredibly difficult problem that algebraist's are extremely interested in. We will focus on more doable cases. There is a cool result that characterizes which rational numbers could appear as roots of a polynomial, it's called the rational roots theorem.
\ \\ \
\ \\ \
\noindent The \textbf{rational roots theorem} says that $(x-\frac ab)$, where
	$\frac ab$ is a fraction in lowest terms, divides a polynomial
	$a_nx^n+\dots+a_1x+a_0$ of degree $n$ precsiely when $b$ divides $a_n$
	and $a$ divides $a_0$. Why might this be the case?
\ \\ \
\begin{Prob}
What are the possible rational roots of the polynomial $x^3-6 x^2+11 x-6$?
\end{Prob}
\ \\ \

\begin{Prob}
Factor the polynomial $x^3-8x^2+21x-18$.
\end{Prob}

\subsection*{The Discriminant}
The discriminant of a degree two polynomial $f(x) = ax^2+bx+c$ is $\Delta = b^2 - 4ac$. We want to think about what the discriminant tells us about the polynomial $f(x)$.
\begin{enumerate}
\item What can we say about $f(x)$ if $\Delta \geq 0$, i.e. how can we write $f(x)$? (Quadratic formula)
\item What happens if $\Delta < 0$? What does this tell us about $f(x)$? Remember we are always working over the real numbers.
\item Can we factor $x^2 -2x + 1$ in the real numbers?
\item How about $x^2 +2x + 7$? Can we factor it in the real numbers? Why or why not?
\end{enumerate}

\section*{Partial fractions}
\subsection*{Heaviside cover-up method}
A rational function $f(x)$ is \textbf{really simple} if it has a proper
	rational expression $f(x)=\dfrac{p(x)}{q(x)}$ such that every root of $q(x)$ has
	order $1$ and $q(x)$ has no factors $x^2+bx+c$ with $b^2-4a<0$. Every really simple rational function has a \textbf{partial fraction
	decomposition}:
	\[
		f(x)=\dfrac{p(x)}{q(x)}=\dfrac{a_{n-1}x^{n-1}+\dots+a_1x+a_0}{(x-r_1)(x-r_2)\cdots(x-r_n)}
		=
		\dfrac{A_1}{x-r_1}+\dfrac{A_2}{x-r_2}+\dots+\dfrac{A_n}{x-r_n}
	\]
Think about why this is the case. This is most easily found using the \textbf{Heaviside cover-up method}:
{\footnotesize
\begin{align*}
	\dfrac{p(x)}{(x-r_1)\cdots(x-r_{k-1})(x-r_k)(x-r_{k+1})\cdots(x-r_n)}
		&=
		\dfrac{A_1}{x-r_1}+\cdots+\dfrac{A_{k-1}}{x-r_{k-1}}+\dfrac{A_k}{x-r_k}+\dfrac{A_{k+1}}{x-r_{k+1}}+\cdots+\dfrac{A_n}{x-r_n}\\
		\dfrac{p(x)}{(x-r_1)\cdots(x-r_{k-1})(x-r_{k+1})\cdots(x-r_n)}
			&=\dfrac{A_1(x-r_k)}{x-r_1}+\cdots+\dfrac{A_{k-1}(x-r_k)}{x-r_{k-1}}+A_k+\dfrac{A_{k+1}(x-r_k)}{x-r_{k+1}}+\cdots+\dfrac{A_n(x-r_k)}{(x-r_n)}\\
		\dfrac{p(r_k)}{(r_k-r_1)\cdots(r_k-r_{k-1})(r_k-r_{k+1})\cdots(r_k-r_n)}&=0+\cdots+0+A_k+0+\cdots+0
\end{align*}}

\noindent So to find the constant above $x-r_k$ in our partial fraction decomposition we just need to plug $r_k$ into the original polynomial and divide by the differences of $r_k$ with all the other roots. Let's see this in practice.
\ \\ \
\begin{Prob}
	Use long division and a partial fraction decomposition to find the
		anti-derivative 
		\[
		\int\dfrac{x^2+x+1}{x^2+2x-3} \; dx.
		\]
\end{Prob}

\begin{Prob}
Find the partial fraction decomposition of each of the following:

(a) $\dfrac1{x^3-x}$
\quad
(b) $\dfrac{x+2}{x^3-x}$
\quad
(c) $\dfrac{x^2+1}{x^3+4x^2+x-6}$
\quad
(d) $\dfrac{2x^2+2x+2}{x^3+4x^2+x-6}$
\end{Prob}
\ \\ \
\begin{Prob}
Integrate all of the expressions in Problem 4 with respect to $x$ using your partial fraction decomposition.
\end{Prob}

\section*{More on the Seven Bridges of K\"onigsberg}
The city of K\"onisberg in Prussia (not Kaliningrad, Russia) was set on both sides of the Pregel River, and included two large islands - Kneiphof and Lomse - which were connected to each other, or to the two mainland portions of the city, by seven bridges. The problem was to devise a walk through the city that would cross each of those bridges once and only once.
\ \\ \
\ \\ \
\noindent Note that one cannot reach an island or mainland via something that is not a bridge or access any bridge without crossing to its other end. In 1736 Euler was able to prove that this problem has no solution. This is harder than finding a solution because you have to show that no possible path will work rather than just giving a path that does. My hint was to consider the picture of the bridges as a {\bf graph}:
\[
\begin{tikzcd}
&\bullet \ar[ld, bend right, dash] \ar[ld, bend left, dash] \ar[rd, dash]&\\
\bullet \ar[rr, dash] \ar[rd, bend right, dash] \ar[rd, bend left, dash] &&\bullet \ar[ld, dash]\\
&\bullet&
\end{tikzcd}
\]
Where each dot, called a {\bf vertex}, represents a land mass and each line, called an {\bf edge}, represents a bridge. The goal is to start at some vertex and walk along each edge exactly once. After Euler's solution to this problem, such walks through graphs became known as {\bf Eulerian Trails}. So what Euler proved is that there is no Eulerian Trail in this graph. He actually proved something more general. He proved exactly when an Eulerian Trail could exist in a graph and showed that this graph does not have the necessary and sufficient condition. I challenge you to think about what this necessary and sufficient condition might be.



\end{document}